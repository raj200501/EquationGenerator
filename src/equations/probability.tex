% This file contains the LaTeX formatted equations for probability

\section*{Probability}

\input{../../data/equations_probability.txt}
